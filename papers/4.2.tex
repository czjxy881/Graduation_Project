\ifx\allfile\undefined
\documentclass{article}
\usepackage{amsmath}%换行
\usepackage{ctex}
\usepackage{indentfirst} %首行缩进
\begin{document}
\section{Hash函数实现快速图相似性搜索}
\else
\subsection{Hash函数实现快速图相似性搜索}
\fi
\subsection{索引构建}
\subsection{\emph{K-NNs}查询过程}
为了获得对于给定图的\emph{K-NNs},我们需要计算它和数据库中其他图的距离。以下是我们定义的利用核函数进行两图距离度量的函数
\begin{equation}
\begin{split}
    &d(G,G')=\sqrt{\|\phi(G)-\phi(G')\|_{2}^{2}}\\
              &=\sqrt{\langle\phi(G)-\phi(G'),\phi(G)-\phi(G')\rangle}\\
              &=\sqrt{\langle\phi(G),\phi(G)\rangle+\langle\phi(G'),\phi(G')\rangle-2\langle\phi(G),\phi(G')\rangle}\\
              &=\sqrt{k_{m}(G,G)+k_{m}(G',G')-2k_{m}(G,G')}
\end{split}
\end{equation}

公式中$k_{m}(G,G)$代表图$G$和其本身的核函数值,$k_{m}(G',G')$是图$G'$及其本身的值,$k_{m}(G,G')$就是图$G$和$G'$的。在后文中,我们会介绍该如何计算这些值。

尽管在将查询图节点哈希到哈希表时,我们可以得到核函数
\begin{equation}
k_{m}(G,G')=\sum_{v\in G',u\in simi(v)}K(\Gamma^{h}(u),\Gamma^{h}(v))
\end{equation}


\ifx\allfiles\undefined
\end{document}
\fi