\ifx\allfiles\undefined
\documentclass{XDBAthesis}
\def\pictures{}
\begin{document}
\else
\fi
\chapter{总结与展望}
\label{chap:future}
本章作为全文的最后一章,将对本文的所述内容进行总结,并对下一步的研究工作给出一些意见。
\section{本文总结}
数据的迅速增长,在给人们带来便捷信息的同时,也带来了一个巨大的挑战——面对日益复杂的数据,传统的查询方法不再有效,无法快速检索出相关联数据。图作为一种通用的数据结构,能对复杂数据很好建模。由此图搜索技术日益成为一个热点课题,本文在前人的基础上,对图的精确搜索和相似性性搜索做了一定深入的研究。主要工作如下:
\begin{enumerate}
    \item 本文首先介绍了下图的基本定义,存储方法,图查询类型和一些经典的查询算法。
    \item 随后,我们对于图精确搜索提出了一种基于二次哈希开链法的搜索算法,有效避免了传统哈希方法冲突频发的问题,加速了整个查询过程。本文中,我们完整得介绍了这种算法,从数据库建库,到查询剪枝,直最后的子图同构检测,并通过实验证明了此算法确实可以加快整个查询过程。
    \item 然后,对于图相似性搜索我们提出了一种基于节点相似度的搜索算法。本算法与G-Hash算法大致相同,但重新定义了其核心部分——图相似度度量方法,使得编码复杂度大大降低,但同时效率又不低于G-Hash算法。本文中我们除了详细介绍了此算法原理,还给出了算法设计类图,并通过实验证明了此算法完全符合我们预期的目标——运行效率不低于G-Hash,甚至在某些情况下略好于G-Hash,但是编码难度大大下降。
\end{enumerate}

综上所述,本文对子图搜索方面做了一定基础性的研究,在认真研究了前人的经典算法基础上,进行了一些新的探索。
\section{不足与展望}
但是,本文提出的两种新算法也只是在传统算法上的一个局部优化,并没有脱离传统算法的框架,因此效率上并没有显著的提升。图数据查询仍是图数据管理中的一个重要领域,还有许多方向可以研究。在进一步的研究中可以从以下几个方面入手:
\begin{enumerate}
    \item 目前,时时刻刻产生着大量的图数据信息,如何在图数据库中有效存储,在内存中以何种结构存储图数据,如何用固定的磁盘页面存储不同规模尺寸的图数据,又如何对图数据进行压缩表示,这些基本的物理存储问题将直接决定I/O操作的用时。而在实际中对图的存取又是非常频繁的,所以如果能解决存储问题,那么必然能提高整体查询效率。
    \item 除了物理存储,逻辑索引显得更为重要,如何高效索引,如何进行维护和更新,这些都是亟待解决的问题。特别是图索引的维护与更新,由于目前大量算法都处于理论阶段而并非实际运用,所以都没有涉及动态维护与更新。但是随着图数据越来越为重要,实际中对图的应用也日益增多。如果能解决这个问题,那么对图从实验室走到实际运用中将会是个强助力。
    \item 图查询的计算复杂度通常能达到指数级及以上,单线程的运算速度就成了图查询速度的瓶颈。因此考虑利用CPU/GPU异构并行或者多处理器并行,多机并行等也是一个不错的研究课题。
\end{enumerate}


\ifx\allfiles\undefined
%\bibliographystyle{unsrt}
\bibliography{main}
\end{document}
\fi