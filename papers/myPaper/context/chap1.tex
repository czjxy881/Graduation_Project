\ifx\allfiles\undefined
\documentclass{XDBAthesis}
\begin{document}
\else
\fi
\chapter{引言}
\label{chap:introduction}
\section{研究背景与意义}
随着科学技术的进一步发展,我们正逐步从\emph{信息时代}走入\emph{数据时代}\cite{BigData},全球的数据量正在以一种前所未有的方式增长着。数据的迅速增长,在给人们带来便捷信息的同时,也带来了一个巨大的挑战——面对日益复杂的数据,传统的查询方法不再有效,无法快速检索出相关联数据。面对大量有意义的数据,无奈于查询手段的限制,只能将其简化再进行处理。现在的大数据现状就好似守着一座金山,却不知如何开采。为了进一步挖掘有效信息,加速查询速度,提高信息价值,各种数据查询技术便应运而生。

其中最为热门的就是图数据库的查询。图作为计算机科学中的一个数据结构,其数据表达能力较强,可以很好得表示

\ifx\allfiles\undefined
\renewcommand\refname{参考文献}
\bibliographystyle{unsrt}
\bibliography{main}
\end{document}
\fi