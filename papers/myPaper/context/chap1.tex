\ifx\allfiles\undefined
\documentclass{XDBAthesis}
\begin{document}
\else
\fi
\chapter{绪论}
\label{chap:introduction}
\section{研究背景与意义}
随着科学技术的进一步发展,我们正逐步从\emph{信息时代}走入\emph{数据时代}\cite{BigData},全球的数据量正在以一种前所未有的方式增长着。统计表明,全球数据几乎每两年翻一番\cite{dataincrease}。数据的迅速增长,在给人们带来便捷信息的同时,也带来了一个巨大的挑战——面对日益复杂的数据,传统的查询方法不再有效,无法快速检索出相关联数据。面对大量有意义的数据,无奈于查询手段的限制,只能将其简化再进行处理。现在的大数据现状就好似守着一座金山,却不知如何开采。为了进一步挖掘有效信息,加速查询速度,提高信息价值,各种数据查询技术便应运而生。

其中最为热门的就是图数据库的查询。图作为计算机科学中的一个数据结构,其数据表达能力较强,可以很好得表示各种关系特征,拓扑结构等,在许多领域都有应用。

举例而言,图数据在以下领域都发挥着显著的作用:
\begin{enumerate}
    \item 在分子化学\cite{yp}与生物\cite{yy}领域,图可以很好得表示分子结构,用节点代表分子,节点属性为分子属性,边代表分子之间的化学键,边长等可以代表化学键键值等。利用图模型建立分子结构数据库,就可以利用图查询技术快速寻找相似的分子结构,来查询具有特定功能的分子集,如$DAYLIGHT$系统\cite{daylight}作为一个商用的分子化合物数据库已经在业界被广泛使用。
    \item 在地理信息系统领域\cite{y6,y7,y8},利用图模型可以完整的表示出各实体之间的关系特征,然后利用这些特征实现进行许多功能,像拓扑关系建模,污染网络绘制,最短路查询\cite{y8} 等。
    \item 在软件工程中,也可以利用图对程序代码进行建模,获得程序调用图,类关系图等,然后利用图相似性搜索可以进行易发故障判定或者代码剽窃检测\cite{copy}等。
    \item 在社会生活中,我们可以利用图结构对人际关系进行建模,进行社群侦查,行为预测等,甚至可以利用其进行犯罪团伙检测\cite{y10}。
    \item 在Web领域,利用图结构进行XML文件分析,新闻的聚类分析等早被广泛使用。
\end{enumerate}

除此之外,图数据还在像文献查重\cite{y9},专家推荐系统等众多领域发挥着重要作用。

由此可见,图数据有着广泛的应用前景。因此如何对图数据进行有效管理与使用,愈加成为数据管理领域一个重大挑战。而如何快速检索图结构则是图数据管理中最为重要的一个问题,已备受研究人员的关注。

图查询种类很多,包括子图查询,超图查询,精确查询,近似查询等\cite{g13}。子图查询是给定查询图$q$在图数据集中找到所有包括$q$的数据图。子图查询是目前图查询中运用最为广泛的,也是研究最久的一种查询类型。如在生物信息学中\cite{g13},当我们知道某一蛋白质分子结构$CA$是HIV病毒中的活跃成分,那就可以通过子图查询,将CA作为查询图,找出所有包含改分子结构的大型分子。

\section{本文的研究目的与内容}
\ifx\allfiles\undefined
\renewcommand\refname{参考文献}
%\bibliographystyle{unsrt}
\bibliography{main}
\end{document}
\fi