\ifx\allfiles\undefined
\documentclass{article}
\usepackage{amsmath}%换行
\usepackage{ctex}
\usepackage{tikz}
\usepackage{subfigure}
\usepackage{graphicx}
\usepackage{caption}
\usepackage{amsfonts, amssymb}%数学字体

\usepackage{indentfirst} %首行缩进
\makeatletter
\newcommand\figcaption{\def\@captype{figure}\caption} 
\newcommand\tabcaption{\def\@captype{table}\caption}
\renewcommand\figurename{图表} 
\makeatother
\usepackage{float}

\begin{document}

\else

\fi
\section{引言}
目前,形如集合,序列,树和图等结构化数据给像高效存储,索引,部分查询(如子图/超图搜索)和相似性搜索之类的传统数据管理领域提出了一个巨大的挑战。随着图数据库的迅速发展,在图数据库中的\emph{图相似性搜索}成为了一个日益重要的研究课题。图相似性搜索已经多个领域进行了应用,例如化学,分子信息学,传感器网络管理,社交网络管理,XML文档等等。在化学与制药领域,每天会产生大量的化学分子结构数据。一旦一个新的化学结构被合成后,这个结构的特性就可能通过查询已知的分子结构,通过其特性来预测。大规模图数据中的相似性搜索

\ifx\allfiles\undefined
%\renewcommand\refname{参考文献}
%\bibliographystyle{unsrt}
%\bibliography{G-Hash翻译}
\end{document}
\fi