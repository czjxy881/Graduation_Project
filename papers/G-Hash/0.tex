\ifx\allfiles\undefined
\documentclass{article}
\usepackage{amsmath}%换行
\usepackage{ctex}
\usepackage{tikz}
\usepackage{subfigure}
\usepackage{graphicx}
\usepackage{caption}
\usepackage{amsfonts, amssymb}%数学字体

\usepackage{indentfirst} %首行缩进
\makeatletter
\newcommand\figcaption{\def\@captype{figure}\caption} 
\newcommand\tabcaption{\def\@captype{table}\caption}
\renewcommand\figurename{图表} 
\makeatother
\usepackage{float}

\begin{document}

\else

\fi
\begin{abstract}
    目前,集合,列表,树和图之类的结构化数据给数据管理这一基础领域提出了一个严峻的挑战。如何对图数据有效存储,索引,搜索都面临着一系列问题。随着图数据库的快速增多,图相似性搜索成为了一个热门的话题。图相似性搜索在多个领域都有应用,像化学结构,分子结构,传感器网络,XML文档等等。
    
    现在大多的图搜索算法都是为了解决精确搜索的,即找到一组包含精确查询图的图集合。因此,我们不能直接利用这些算法进行相似性搜索。在数据挖掘和机器学习领域有很多图核函数来判断图之间的固有相似度,但在图相似性搜索上却很罕见。因为尽管图核函数对于监督学习领域的准确预测和分类模型效果很好,但是用于图相似度查询则会存在两个关键问题:(i)计算复杂度非常高(ii)在图搜索上的应用复杂度也是非平凡的。
    
    而本文中,我们打算利用图核函数进行相似性搜索。为此,我们提出了两个关键概念(i)一种新的图相似度度量方法(ii)一种新的图数据索引策略。我们以每个节点和其邻接节点特征为相似性度量依据,并利用哈希表来进行高效存储和快速查询。我们这种利用图核函数抓住图相似性的本质特征,并利用哈希表加速查询过程的新算法,我们称之为G-Hash。本文中我们详细介绍了这种方法,并在大规模的化学结构数据库上进行了实验。结果表明G-Hash在k个最相近邻居问题上已经达到了业界的最高水平,更重要的是,我们这种新的相似性度量方法和索引结构比现有的算法(C-tree,gIndex,GraphGrep等)具有更小的索引尺寸和更快的查询速度。
    
    \textbf{关键词:图相似性搜索,图分类,哈希,图核,k-NNs查询}
\end{abstract}

\ifx\allfiles\undefined
%\renewcommand\refname{参考文献}
%\bibliographystyle{unsrt}
%\bibliography{G-Hash翻译}
\end{document}
\fi