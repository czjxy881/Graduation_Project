\ifx\allfiles\undefined
\documentclass{article}
\usepackage{amsmath}%换行
\usepackage{ctex}
\usepackage{tikz}
\usepackage{subfigure}
\usepackage{graphicx}
\usepackage{caption}
\usepackage{amsfonts, amssymb}%数学字体

\usepackage{indentfirst} %首行缩进
\makeatletter
\newcommand\figcaption{\def\@captype{figure}\caption} 
\newcommand\tabcaption{\def\@captype{table}\caption}
\renewcommand\figurename{图表} 
\makeatother
\usepackage{float}

\begin{document}

\else

\fi
\section{基于哈希的快速图相似性搜索}
正如之前所说,现在的图查询算法查询时间很快但是没有好的相似性度量方法。核函数可以提供一个好的相似性度量方法,但是核函数的矩阵运算需要大量的时间,所以我们很难直接利用其来建立索引。为了解决这个问题,我们提出了一种新的算法,G-Hash。现在的算法常常关注速度或精度的一种,而G-Hash在两者效果均很好。我们利用小波图匹配核(WA)来定义相似性,并用Hash表作为索引结构来加速图相似性查询。下面我们先介绍下WA方法。

\ifx\allfiles\undefined
%\renewcommand\refname{参考文献}
%\bibliographystyle{unsrt}
%\bibliography{G-Hash翻译}
\end{document}
\fi