\ifx\allfiles\undefined
\documentclass{article}
\usepackage{amsmath}%换行
\usepackage{ctex}
\usepackage{tikz}
\usepackage{subfigure}
\usepackage{graphicx}
\usepackage{caption}
\usepackage{indentfirst} %首行缩进
\makeatletter
\newcommand\figcaption{\def\@captype{figure}\caption} 
\newcommand\tabcaption{\def\@captype{table}\caption}
\renewcommand\figurename{图表} 
\makeatother
\usepackage{float}


\begin{document}
\section{背景知识}
\else
\subsection{背景知识}
\fi
在我们详细介绍算法细节之前,让我们先了解一下关于图分析计算的所需的基本背景。这章包含(i)图核函数,(ii)图小波分析两部分。
\subsection{图}
一个\emph{标号图}可以被一个有限的节点集合$V$和一个有限的边集合$E\in V\times V$所描述。在大多数应用中,图都是有标号的。这些标号都是从一个标号集选取的,我们用一个标号函数$\lambda :V\cup E\rightarrow \Sigma$来给各个节点和边分配标号。在\emph{标号点图}中只有点有标号,同样,在\emph{标号边图}中只有边有标号。在\emph{全标号图}中,边点都有标号。如果用一种特殊的标号来表示未标号的点和边,那么标号点图和标号边图都可以被看做是全标号图的特殊形式。所以在此文中我们只考虑全标号图来简化问题而又不失一般性。对于标号集$\Sigma$我们并不指定具体结构,可以是一个字段,一个向量,也可以是很简单的是个集合。
以下我们约定,一个图用一个四元组$G=(V,E,\Sigma ,\lambda )$表示,$V,E,\Sigma,\lambda$都如上文所述。如果一个图$G=(V,E,\Sigma,\lambda)$和另一个图$G'=(V',E',\Sigma',\lambda')$有1-1映射的关系$f:V\rightarrow V'$,如对于所有,那么图$G$就是$G'$的\emph{子图},用$G\in G' $表示



\ifx\allfiles\undefined
%\renewcommand\refname{参考文献}
%\bibliographystyle{unsrt}
%\bibliography{G-Hash翻译}
\end{document}
\fi