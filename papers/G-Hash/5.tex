\ifx\allfiles\undefined
\documentclass{article}
\usepackage{amsmath}%换行
\usepackage{ctex}
\usepackage{tikz}
\usepackage{subfigure}
\usepackage{graphicx}
\usepackage{caption}
\usepackage{amsfonts, amssymb}%数学字体

\usepackage{indentfirst} %首行缩进
\makeatletter
\newcommand\figcaption{\def\@captype{figure}\caption} 
\newcommand\tabcaption{\def\@captype{table}\caption}
\renewcommand\figurename{图表} 
\makeatother
\usepackage{float}

\begin{document}

\else

\fi
\section{实验研究}
我们对我们的算法进行了广泛的实验研究来评估其有效性,高效性及扩展性。我们在化学分子结构上测试我们的算法。对于化学结构,节点特征包括数值特征和原子布尔特征。数值特征包括元素种类,原子部分电荷,原子电子亲和势,原子自由电子数目和原子价态等等。布尔特征包括原子是否在供体中,是否在末端碳中,是否在环中,是否为负,是否是轴向的等等。在实验中,我们仅用一个原子特征:元素种类。

我们将我们的方法和小波分配核,C-tree,GraphGrep还有gIndex进行比对。我们的算法,WA算法,GraphGrep和gIndex是基于C++实现的,用g++进行编译。C-tree是用Java实现的,用Sun JDK 1.5.0编译。所有的实验都是在Intel Xeon EM64T 3.2GHz,4G内存,Linux系统这一平台上测试的。

WA,G-Hash,C-tree,GraphGrep和gIndex的参数是这样设置的。对于WA和G-hash,h取2,用\emph{haar}小波函数,对于C-tree,用默认值即将最小子节点数m设为20,最大M设为$2m-1$,用NBM方法进行图映射。对于GraphGrep和gIndex,全部采用默认参数。

\subsection{数据集}
我们选用许多数据集来进行试验。前五个数据集是从从Jorisson/Gilson数据集获得的已有数据。接下来六个是从BindingDB数据集中抽取的,最后一个是NCI/NIH 艾滋病筛选集里的,表1显示了这些数据集和其基本情况。
\subsubsection{Jorissen数据集}
Jorissen数据集主要为一些包含活动的化学结构信息


\ifx\allfiles\undefined
%\renewcommand\refname{参考文献}
%\bibliographystyle{unsrt}
%\bibliography{G-Hash翻译}
\end{document}
\fi