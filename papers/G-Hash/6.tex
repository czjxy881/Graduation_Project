\ifx\allfiles\undefined
\documentclass{article}
\usepackage{amsmath}%换行
\usepackage{ctex}
\usepackage{tikz}
\usepackage{subfigure}
\usepackage{graphicx}
\usepackage{caption}
\usepackage{amsfonts, amssymb}%数学字体

\usepackage{indentfirst} %首行缩进
\makeatletter
\newcommand\figcaption{\def\@captype{figure}\caption} 
\newcommand\tabcaption{\def\@captype{table}\caption}
\renewcommand\figurename{图表} 
\makeatother
\usepackage{float}

\begin{document}

\else

\fi
\chapter{总结与展望}
图作为一种通用的数据结构已被广泛运用于大量领域,像化学分子学,生物信息学等等。大量知名学者当前都专注于数据管理方法和数据挖掘技术。而如何提出一种有效的图相似性搜索方法一直是一个重大的难题,因为现在大量算法都只专注于速度却不关心精确度。为了解决这个问题,我们提出了一种新的图查询算法G-Hash。通过我们的实验研究,我们在构造时间和效率上都明显有优势,这表明G-Hash已经可以支持大规模图数据库搜索。
\chapter{致谢}
感谢H.He和A.K.Singh提供C-tree源代码。感谢D.Shasha,J.T.L.Wang和R.Giugno提供的GraphGrep。感谢X.Yan,P.S.Yu和J.Han提供GIndexy源代码。我们的所有工作均是在KU中心支持下完成的,对他们表示由衷的感谢。
\ifx\allfiles\undefined
%\renewcommand\refname{参考文献}
%\bibliographystyle{unsrt}
%\bibliography{G-Hash翻译}
\end{document}
\fi